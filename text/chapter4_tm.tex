%\begin{savequote}[8cm]
%\textlatin{Neque porro quisquam est qui dolorem ipsum quia dolor sit amet, consectetur, adipisci velit...}

%There is no one who loves pain itself, who seeks after it and wants to have it, simply because it is pain...
%  \qauthor{--- Cicero's \textit{de Finibus Bonorum et Malorum}}
%\end{savequote}

\chapter{\label{ch:1-intro} Topic Modelling and Latent Dirichlet Allocation for Unsupervised Clustering of Bulk ATAC-seq Experiments}

\minitoc

\providecommand{\tightlist}{%
  \setlength{\itemsep}{0pt}\setlength{\parskip}{0pt}}

%\section{Motivation}
\section{Erythropoesis as a well-characterised model system}

Because of the relative sparsity of well characterised MLL-r specific enhancers, we choose to first model a differentiation system which has been well characterised. Erythropoesis is the process by which hematopoetic stem cells first differentiate into progenitor populations before committing to the myeloid lineage and eventually undergoing enucleation and terminal differentition into mature erythrocytes. This system has been extensively studied (cite Ludwig and Corces and the protein one) and a catelog of differentially active transcription factors and associated enhancer elements can be readily derived from the literature. We seek to validate the proposed topic modelling approach on this model system by systematically building up a model of erythropoesis, including sequentially various detractor lineages such as a related differentiation to B cell lymphopoesis, asking whether topics are able to reproducibly recapitulate known dynamics of lineage committment and terminal differentiation. 


\begin{enumerate}
    \item Talk about other tools for performing this task and how this is not actually the same question. We need to be able to differentiate between arbitrary number of cells and cell states, not limited to just lookign at systems where the full differentiaiton hierarchy is available. 
    \item Talk about the methods for fidning these differentially accessible elements between the different stages. 
\end{enumerate}



\subsection{Data preperation and peak calling}

\begin{enumerate}
    \item Where was the data downloaded from? How was it processed?
    \item Peak calling. MaCS3 versus lance-o-tron. What kind of diagnostic features can we see here? 
    \item Number of peaks. 
    \item Peak length distribution
\end{enumerate}A


\section{Results}



\begin{itemize}
    \tightlist
    \item
      LDA is shown to be a viable method for single cell ATAC. Does it also
      work on bulk experiments?
    \item
      LDA is able to find reproducible topic loadings that differentiate
      pseudobulked clusters.
    
      \begin{itemize}
      \tightlist
      \item
        (Pseudobulk cluster topic loadings resemble single cell topic
        loadings)
      \item
        \textbf{Question}: Have a view on what the best data for this
        question would have been * (put this into the discussion)
      \end{itemize}
    \item
      LDA versus differential OCR \textbf{\emph{(think more about this)}}
    \item
      Topics are reproducible across stochastic replicates, but variously
      so.
    
      \begin{itemize}
      \tightlist
      \item
        Dimensionality reduction with and without the not-reproducible
        topics
      \item
        Are there any systematic differences between the reproducible topics
        and the not reproducible ones (i.e.~differences in strengths of
        loadings, distribution across the cells, localisation in a cluster,
        etc.)?
      \end{itemize}
    \item
      Hyper parameter optimisation strategy
    
      \begin{itemize}
      \tightlist
      \item
        Dask application for bayesian optimisation of hyper parameters to
        maximise the LL while integrating over the number of topics selected
      \end{itemize}
    \item
      Methods for isolating regions from region-topic annotations
    
      \begin{itemize}
      \tightlist
      \item
        Compare motifs identified through thresholding, gamma distribution
        threshold, top N approach, etc. Which produces the most reliable set
        of motifs for the various topics?
      \end{itemize}
    \item
      Applicability to a biological system: Erythropoesis
    
      \begin{itemize}
      \tightlist
      \item
        Hyper parameter search using the Dask application
      \item
        Topic loadings are stage specific.
      \item
        Motifs represent biologically relevant transcription factors that
        are active at each stage and recapitulate the dynamics of
        erythropoesis
      \item
        Major topics are consistent across replications and the stages
        always seperate based on topic loadings
      \end{itemize}
    \item
      LDA recapitulates larger scale trends in chromatin accessibility:
    
      \begin{itemize}
      \tightlist
      \item
        Topic distribution in the ENCODE project recapitulates large scale
        trends in accessibility.
      \item
        Dig into a couple of the interesting topics and try to figure out
        what they are doing here.
      \end{itemize}
    \item
      Conclusion: LDA represents an interesting and useful method for
      Unsupervised clustering of different cell types from bulk ATAC-seq
    \end{itemize}
    
    \hypertarget{chapter-5-topic-modelling-prioritises-ocrs-in-mll-af4-recombinant-cells}{%
    \subsection{Chapter 5: Topic Modelling Prioritises OCRs in MLL-AF4
    Recombinant
    Cells}\label{chapter-5-topic-modelling-prioritises-ocrs-in-mll-af4-recombinant-cells}}
    
    \begin{itemize}
    \tightlist
    \item
      \textbf{Describe what the appropriate validation should be here. From
      these results, I would like to see X, Y, Z experimental approaches.}
    \end{itemize}
    
    \textbf{Main question}: \emph{What distinguishes MLL-AF4 from healthy
    cells in terms of open chromatin?}
    
    \begin{itemize}
    \tightlist
    \item
      LoT results in cleaner peak calls for topic modelling in MLL-AF4 cells
    \item
      RS4;11 and SEM can be merged to form a cohesive MLL-AF4 group
    
      \begin{itemize}
      \tightlist
      \item
        Differential accessibility analysis between the two different cell
        types. (What is different, and why is this not important.)
      \end{itemize}
    \item
      Differential peak analysis between MLL-AF4 and BCP (B cell precursors)
      to provide a baseline expectations of the regions we may see
    \item
    \end{itemize}